\documentclass[../thesis.tex]{subfiles}

\begin{document}
\chapter{Pendahuluan}\label{chap:pendahuluan}
\section{Latar Belakang}

%THE IMPORTANCE RIVERSIDE IN GENERAL
Riverside merupakan ruang perkotaan yang harus terus berkembang. Kawasan inilah yang diberkahi dengan karakteristik dan perhatian khusus mengingat pentingnya air sebagai sumber kehidupan kota \citep{shamsuddin2013}.
Pada area laut perkotaan, lomba untuk ruang waterfront, kebutuhan publik untuk mengakses pesisir laut dan mempertahankan biodiversity tepi laut sebagai sumber alami menjadi isu terhangat dalam kebijakan perkotaan \citep{breen1994waterfronts}.
%KEGAGALAN WATERFRONT IN GENERAL
Tidak hanya memberikan sebuah ruang pada suatu lahan di perkotaan, tetapi interaksi antara ruang dan pengguna adalah sangat penting. Pengembangan tepi laut seharusnya yang menciptakan ruang publik yang menyenangkan. Banyak kota yang telah gagal dalam memberikan kesan pada sebuah Waterfront, pandangan orang menjadi sosok yang kotor, padat, bahkan bahaya di sebuah sudut kota. Dalam pandangan \cite{goodwin1999} tepi laut biasa dipersepsikan sebagai sesuatu yang samar dan tercampur dengan area yang diabaikan, pusat komersial dan permukiman.


%PENURUNAN KUALITAS
Seiring perkembangan kota, waterfront seringkali mengamalami kegoyahan dalam perkembangannya. Beberapa hal yang menyebabkan penurunan kualitas  waterfront adalah adanya peningkatan mobilitas yang melewati area tersebut\citep{richarda.lehmann1966}. \cite{ulam2009} menjelaskan kawasan waterfront pada paruh kedua abad ke-20 Waterfront harlem mengalami penurunan kualitas ditandai tempat hanya menjadi tempat memancing, penggunaan narkoba dan prostitusi. Bahkan menerima presepsi yang buruk oleh penduduk, tidak dapat diakses, privatisasi bank, kontaminasi air dan pembukaan jalur tol pararel dengan pesisir \citep{shamsuddin2013}. Menurut \cite{benages2015revisiting} tepi laut mengalami seperti penurunan yang signifikan disebabkan oleh berbagai faktor seperti tekanan \textit{real estate} dan perencanaan yang kurang. Kejadian ini membuat pemerintah kesusahan dalam mengatur pengembangan tepi laut \citep{gripaios1999ports}.
%REDEVELOPMENT
Meskipun demikian, pengembangan waterfront kian meningkat dari abad 19. Misalnya tepi sungai kota Witconsin yang mengalami kegagalan dalam memberikan rasa ruang dan perbedaan kota itu akhirnya berbedah. Dengan konsep `Waterfront Renewal' melibatkan 50 kota di Amerika \citep{richarda.lehmann1966}. Ditambah lagi dengan `waterfront redevelopment' pada pelabuhan Thessaloniki di Yunani. Serta di Liverpool, inggris sekitar tahun 1940 \citep{couch2003city}.

%WATERFRONT RESTORE CITY
Pesisir laut saat ini menjadi fokus pengembangan dalam sebuah kota. Dalam penelitian pelabuhan dan tepi laut industri di Thessaloniki, Yunani. Meskipun dalam keadaan ekonomi terpuruk, mereka masih menyediakan area yang cukup luas untuk bangunan mewah baru yang berdiri di pinggir laut. Pembangunan tersebut kebanyakan berada di pusat kota dan menjadi simbol suatu ekonomi 'sukses' \citep{vayona2011}.

Kota Parepare sendiri telah lama memiliki sejumlah ruang perkotaan di pesisir laut. Beberapa diantaranya adalah kawasan tonrangeng riverside, taman mattirotasi, jalan mattirotasi dan lain-lain. Banyak warga yang berkativitas dengan cara istirahat, piknik dan bahkan memancing. Ini menjadi sebuah gambaran bahwa masyarakat Parepare, selain dari segi ekonomi, mereka sangat membutuhkan interaksi atau aktivitas dalam sebuah ruang publik yang berdekatan dengan air.
%AKTIVITAS PINGGIRLAUT
Jalan pinggir laut(baca: waterfront) telah lama menjadi unggulan penduduk kota Parepare dari anak kecil hingga dewasa. Beberapa tahun yang lalu, banyak orang ke tempat ini untuk melakukan aktivitas rekreasi, olahraga, bahkan bersantai. Kadang mereka datang secara kelompok, berdua, atau sendiri.
Objek makan di pinggir laut menjadi favorit kebanyakan orang. Panorama yang indah dan dilengkapi penjual bubur kacang `ijo' sangat memuaskan siapapun yang berkunjung. Berdasarkan pengamatan, aktivitas inilah yang dominan dalam ruang tersebut. Sehingga keberlangsungan pemanfaatan ruang publik terus berlanjut.

%HUBUNGAN LINGKUNGAN -> WELLBEING
Perencanaan perkotaan memainkan peran yang sangat penting dalam meningkatkan kesehatan dan kesejahteraan manusia \citep{sarkar2017urban}, seperti contoh dengan rekreasi makan di pinggir laut. Menurut \cite{eckstutassociates1986} kebutuhan manusia dalam ruang publik setidaknya berdasarkan berikut ini: kemudahan, keamanan, kenyamanan, keindahan, kegunaan dan menarik. Sehingga kebutuhan-kebutuhan tersebut dapat mendorong aktivitas luar pada waterfront.
\cite{vancauwenberg2018} menyebutkan berpartisipasi dalam aktivitas fisik selama waktu santai dapat melepaskan tekanan\textit{stress}, menambahkan rasa arti dalam hidup, membantu keterbatasan yang disebabkan oleh penyakit kritis dan menghilangkan kegiatan hidup yang negatif. David Mengungkapkan pada \citep{richarda.lehmann1966} bahwa minat dasar manusia dalam fasilitas dan atraksi riverside  menghasilkan tingkat kegembiraan yang tinggi atas pengembangan riverside.
Dalam \citep{adams2013} mengetahui dampak dari kualitas ruang publik terhadap kesejahteraan adalah kebutuhan. Penggunaan dan fungsi menjadi hal terpenting daripada hanya sebuah desain dan kepemilikan. \cite{sairinen2006} menyebutkan waterfront dapat pula menjadi \textit{image of place} dan sebagai sumber daya alami atau greenspace pada struktur perkotaan.

%PENGETAHUAN DETAIL KUALITAS -> ACTIVITY
Fisik lingkungan yang dirancang memadai dapat memicu pengguna untuk menjalankan gaya hidup yang aktif dan dinamis\citep{chang2020}. Sehingga pendekatan lingkungan binaan memiliki alasan yang kuat untuk mendorong aktivitas diluar.
%MASALAH
Beberapa waktu lalu, Waterfront Parepare ini mengalami redevelopment setelah pergantian walikota. Dimana terdapat perubahan-perubahan signifikan secara fitur fisik \textit{(physical feature)}. Tentunya hal ini membuat perubahan kualitas ruang publik itu terhadap gaya hidup atau aktivitas didalamnya. Hal ini menjadi latar belakang penelitian ini yaitu mengetahui hubungan kualitas lingkungan binaan waterfront terhadap jenis aktivitas luar\textit{(outdoor activity)} di Jl. Pinggir Laut di Kota Parepare.




\section{Rumusan Masalah}
Berdasarkan latar belakang permasalahan yang diangkat, yaitu adanya \textit{waterfront redevelopment} pada kawasan Jl. Pinggir Laut. Dimana mengubah besar kecilnya kualitas waterfront tersebut. Melalui studi yang dilakukan ternyata sangat penting dalam mengetahui dampak penggunaan dan fungsi waterfront dibanding pengakuan keberadaan dan desain semata. Oleh karena perlunya pengetahuan terhadap dampak \textit{usability} dan \textit{functionality} maka diambil aktivitas luar \textit{(outdoor activity)} sebagai variabel yang dipengaruhi (\textit{dependent}). Dari pemahaman tersebut penelitian ini merumuskan rumusan penelitian sebagai berikut:


\begin{itemize}
	\item Bagaimana kualitas fitur fisik binaan setelah \textit{waterfront development}?
	\item Apa saja aktvitas luar (\textit{outdoor activity}) yang ada di tepi laut Jl. Pinggir Laut?
	\item Apakah ada dan bagaimana hubungan fitur fisik binaan terhadap aktivitas luar?
\end{itemize}

\section{Tujuan Penelitian}
Penelitian ini bertujuan untuk mengetahui kebenaran adanya pengaruh dan bagaimana pengaruh kualitas urban waterfront setelah redevelopment khususnya fitur fisik dan sosial yang ada terhadap sejumlah aktivitas luar. Sehingga muncul indikator atau sub-variabel yang signifikan terhadap munculnya aktivitas luar.


\section{Manfaat Penelitian}
\begin{itemize}
	\item Penilitian ini diharapkan menjadi acuan dalam skala yang berbeda, untuk penjabaran kebijakan publik dan pengadaan strategi perencanaan dimana tepi laut dipertimbangkan sebagai area yang mempengaruhi kota secara menyeluruh.
\end{itemize}

\section{Sistematika Penulisan}
Berikut ini adalah sistematika penulisan yang digunakan pada penelitian dimensi kenyamanan pada Waterfront Development:
\begin{itemize}
	\item Bab 1 : Pendahuluan\\
Bab terdiri dari latar belakang permasalahan, perumusan masalah, tujuan penelitian, manfaat penelitian, dan sistematika penulisan.
	\item Bab 2 : Tinjauan Pustaka\\
Bab ini terdiri dari landasan teori yang digunakan untuk memperkuat penemuan masalah, penelitian terdahulu dan kerangka pemikiran.
	\item Bab 3 : Metodologi Penelitian\\
Bab ini terdiri dari penjelasan variabel dan jenis paradigma yang digunakan untuk mencapai penemuan sesuai rumusan masalah, populasi, sampel, dan cara pengumpulan data.
	\item Bab 4 : Hasil dan Pembahasan\\
Bab ini terdiri dari pembahasan mengenai hasil - hasil penelitian yang berupa data-data yang didapatkan, dengan melakukan pengolahan terhadap indikator-indikator kenyamanan. Setelah pengelolahan bahan-bahan tersebut, analisis diperlukan untuk menemukan penemuan penelitian. Analisis diarahkan untuk menjawab rumusan masalah.
	\item Bab V : Kesimpulan\\
Bab terakhir terdiri dari kesimpulan yang didapatkan dari analisis terhadap permasalahan yang terdapat pada penelitian ini, sehingga penemuan bersama saran-saran dari penelusi dapat menghasilkan apa yang diinginkan.


\end{itemize}
\section{Alur Pikir}

\begin{figure}[hp]
\centering
\begin{tikzpicture}[node distance=2cm]
\node (ltr) [startstop] {Latar Belakang};

\node (rum) [startstop, right of=ltr, xshift=2cm] {Perumusan Masalah};

\node (tuj) [startstop, below of=rum, yshift=0.5cm] {Tujuan Penelitian};


\node (pus) [startstop, below of=tuj, yshift=0.5cm] {Studi Pustaka};


\node (kaj) [startstop, below of=pus, text width=3.5cm, xshift= -4cm, yshift=.5cm] {
	\textbf{Kajian Teori}\\ - Fitur binaan\\ - Aktivitas Luar
};


\node (kaj2) [startstop, below of=pus, text width=3.5cm, xshift= 4cm, yshift=.5cm] {
	\textbf{Gambaran Objek}\\ Fitur Binaan dan Aktivitas Luar Jl. Pinggir Laut
};


\node (hip) [startstop, below of=pus, yshift=-.5cm] {Hipotesa};


\node (met) [startstop, below of=hip, yshift=-.75cm, text width=7cm] {
	\textbf{Metode Peneltian}\\ Menggunakan Metode penelitian Kuantitatif Rasionalistik

	\textbf{Variabel}\\
	- Bebas : Fitur Binaan\\
	- Terikat : Aktivitas Luar\\

	\textbf{Sumber data}: Observasi dan Kuesioner
};

\node (ana) [startstop, below of=met, text width=8cm, yshift=-2cm] {
		\textbf{Analisis Data Statistik}\\ Penelitian ini menggunakan metode statika berupa uji regresi guna mengetahui pengaruh variabel fitur binaan terhadap variabel aktivitas luar.
};

\node (tem) [startstop, below of=ana, yshift=-.25cm] {Temuan Penelitian};

\node (kes) [startstop, below of=tem, yshift=.6cm] {Kesimpulan dan Rekomendasi};

\draw [arrow] (ltr) -- (rum);
\draw [arrow] (rum) -- (tuj);
\draw [arrow] (tuj) -- (pus);

\draw [arrow] (pus) -| (kaj);
\draw [arrow] (pus) -| (kaj2);

\draw [doublearrow] (kaj) -- (kaj2);

\draw [arrow] (kaj) |- (met);
\draw [dotted] (kaj) |- (hip);

\draw [arrow] (kaj2) |- (met);
\draw [dotted] (kaj2) |- (hip);

\draw [arrow] (met) -- (ana);
\draw [arrow] (ana) -- (tem);

\draw [arrow] (tem) -- (kes);

\end{tikzpicture}
\caption{Alur Pikir}
\end{figure}

\newpage

%\onlyinsubfile{\biblio}
\end{document}
