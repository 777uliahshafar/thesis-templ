\documentclass[../thesis.tex]{subfiles}

\begin{document}
\chapter{Metodologi Penelitian}\label{chap:method}

\section{Metode dan Jenis Penelitian}

%kebenaran fitur binaan
\textit{Environmental Evaluation} menjadi hal yang populer pada penelitian berkaitan dengan kualitas yang ada pada taman.  Peneliti mengambil karakteristik dari sebuah tempat untuk dijadikan bahan evaluasi. Ada variabel yang mempengaruhi atau independen serta ada variabel yang dipengaruhi atau variabel dependen. Beberapa peneliti seperti pada ref menggunakana EAPRS sebagai alat untuk mengukur kualitas fitur binaan lingkungan.

%metode
Penelitian ini menggunakan paradigma kuantitatif deskriptif. Bertujuan untuk menggambarkan kondisi secara nyata terhadap hubungan karakteristik fisikdan partisipasi aktivitas fisik pada sebuah tempat.
Penelitian kuantitaif bertujuan untuk menganalisis suatu fenomena sosial dan kemanusian melaluidata berupa angka yang dapat diolah untuk dianalisis dengan tujuan mendapatkan suatu kebenaran dari fenomena tersebut.
Sedangkan rasionalistik merupakan landasan yang berasal dari pemahaman intelektual manusia. Artinya manusia membangun sebuah pemahaman berdasakan argumentasi secara logik dan bukan berdasarkan pengalaman emperi tetapi pada pemaknaan emperi\citep{muhadjir1996metodologi}.

\section{Teknik Pengumpulan data}
Untuk memperoleh data yang diperlukan , maka cara pengumpulan data yang dilakukan penulis diantaranya: observasi, wawancara, dan survei kuesioner serta studi pustaka.

\section{Teknik Penentuan Populasi dan Sampel}

Populasi pada penelitian ini adalah individu atau sekolompok orang yang melakukan aktivitas pada waterfront Jl. Pinggir Laut. Orang yang sedang beraktivitas tersebut diyakini dapat membagi pengetahuan atau pemahaman empiri suatu lokasi berdasarkan kognitif dan perasaannya.

Untuk memperkecil kompleksitas dari penelitian ini, maka ditentukan sampel dari populasi yang ada. Young adult adalah sampel yang lebih dominan dan dapat  dipercaya dalam penggunaan area publik semacam ini.

%\onlyinsubfile{\biblio}
\end{document}
